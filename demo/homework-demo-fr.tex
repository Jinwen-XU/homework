\documentclass[11pt,
  % twoside,
  logo = {example-image},
  % logo height = 1cm,  % logo width = 2cm,
  title in boldface,
  % title in sffamily,
  theorem in new line,
  % remove qed,
  colored solution,
]{homework}

%% For highlighting the code in this document
\usepackage{newverbs}
\newverbcommand{\cverb}{\color{red!50!orange}}{}

%% Below is the main document


\UseLanguage{French}


\title{Le sujet, semaine 1}
\author{Prénom NOM}
% \date{\today, Lieu}
% \date{\today[only-year-month], Lieu}
\date{\TheDate{2023-12-25}, Lieu}


\begin{document}


\textcolor{gray!55}{(Si vous souhaitez écrire la réponse directement...)}

\begin{problem}
    Voici la solution / la preuve.
\end{problem}


\bigskip\textcolor{gray!55}{(Si vous souhaitez énoncer le problème puis écrire votre réponse...)}

\begin{problem}[Description brève]
    Vous pouvez également énoncer le problème ici...
\end{problem}

\begin{solution}
    ... puis écrire la solution ici...
\end{solution}

\bigskip\textcolor{gray!55}{(Si vous préférez \textquote{Preuve} au lieu de \textquote{Solution}...)}

\begin{solution}[Preuve]
    ... ou une preuve comme ça...
    \begin{lemma}[Vous pouvez écrire ici quelques descriptions]\label{lem}
        Un résultat auxiliaire.
    \end{lemma}
    \begin{proof}
        La preuve \cref[de]{lem}, où l'on utilise la formule suivante (notez l'utilisation de \cverb|\qedhere|) :
        \[
            \infty = \infty + 1.
            \qedhere % pour placer le symbole Q.E.D. au bon endroit
        \]
    \end{proof}
    \begin{fact}[Cet énoncé ne requiert aucune preuve]
        \proofless
        Utilisez \cverb|\proofless| pour transformer la boîte creuse marquant la fin d'un environnement de type théorème en une boîte solide.
    \end{fact}
    ... et les étapes restantes...
\end{solution}


\bigskip\textcolor{gray!55}{(Vous pouvez également écrire \texttt{answer} au lieu de \texttt{solution} si vous le souhaitez...)}

\begin{answer}
    L'utilisation de l'environnement \verb|answer| est exactement le même que celui de \verb|solution|.
\end{answer}


\enlargethispage*{2\baselineskip}


\bigskip\textcolor{gray!55}{(Si vous préférez le style classique...)}

\begin{proof}
    L'environnement habituel \verb|proof| fonctionne également.
\end{proof}


\bigskip\textcolor{gray!55}{(Si vous souhaitez répondre individuellement à chaque sous-question d'un problème...)}

\begin{problem}[Un problème avec de nombreuses sous-questions]
    \begin{enumerate}
        \item La première question.

        \begin{solution}
            La réponse à la première question.
        \end{solution}

        \item La deuxième question.

        \begin{enumerate}
            \item La première sous-question.

            \begin{solution}
                La réponse à la première sous-question.
            \end{solution}

            \item La deuxième sous-question.

            \begin{solution}
                La réponse à la deuxième sous-question.
            \end{solution}

        \end{enumerate}

        \item La troisième question.

        \begin{solution}
            La réponse à la troisième question.
        \end{solution}

    \end{enumerate}
    Utilisez \cverb|\noqed| (ou \cverb|\noQED|) à la fin pour supprimer le Q.E.D. symbole qui marque la fin du problème actuel.
    \noQED
\end{problem}


\bigskip\textcolor{gray!55}{(S'il y a une question que vous n'arrivez pas à résoudre pour le moment...)}

\DNF<la description>


\end{document}
