\documentclass[11pt,
  % twoside,
  logo = {example-image},
  % logo height = 1cm,  % logo width = 2cm,
  title in boldface,
  % title in sffamily,
  theorem in new line,
  % remove qed,
  colored solution,
]{homework}

%% For highlighting the code in this document
\usepackage{newverbs}
\newverbcommand{\cverb}{\color{red!50!orange}}{}

%% Below is the main document


\UseLanguage{Traditional Chinese}


\title{學科名稱,第一周}
\author{作者}
% \date{\today,地點}
% \date{\today[only-year-month],地點}
\date{\TheDate{2024-01-01},地點}


\begin{document}


\textcolor{gray!55}{(如果你打算直接寫出解答…)}

\begin{problem}
    這裡是解答/證明。
\end{problem}


\bigskip\textcolor{gray!55}{(如果你打算先陳述問題後再寫出解答…)}

\begin{problem}[問題簡介]
    你可以把問題陳述在這裡…
\end{problem}

\begin{solution}
    …然後把解答寫在這裡…
\end{solution}

\bigskip\textcolor{gray!55}{(如果比起\textquote{解},你更希望寫\textquote{證明}…)}

\begin{solution}[證明]
    …或者寫一個這樣的證明…
    \begin{lemma}[你可以在這裡寫一些註釋]\label{lem}
        一些輔助結果。
    \end{lemma}
    \begin{proof}
        這是\cref{lem}的證明,其中用到下面的公式(注意使用 \cverb|\qedhere|):
        \[
            \infty = \infty + 1
            \makebox[0pt][l]{\,。} % 句尾的句號
            \qedhere               % 用來把 Q.E.D. 符號放在正確的位置
        \]
    \end{proof}
    \begin{fact}[這個結論無須證明]
        \proofless
        使用 \cverb|\proofless| 可以將標記定理結束的空心方框變成實心方框。
    \end{fact}
    …剩餘的步驟…
\end{solution}

\bigskip\textcolor{gray!55}{(寫\textquote{答}也是可以的…)}

\begin{answer}
    \verb|answer| 環境的用法和 \verb|solution| 是完全相同的。
\end{answer}


\enlargethispage*{\baselineskip}


\bigskip\textcolor{gray!55}{(或者如果你更喜歡傳統的證明樣式…)}

\begin{proof}
    \verb|proof| 環境仍可使用。
\end{proof}


\bigskip\textcolor{gray!55}{(如果你想為一個大問題中的每個小問題分別撰寫解答…)}

\begin{problem}[一個由許多小問題構成的大問題]
    \begin{enumerate}
        \item 第一個問題。

        \begin{solution}
            第一個問題的解答。
        \end{solution}

        \item 第二個問題。

        \begin{enumerate}
            \item 第一個小問。

            \begin{solution}
                第一個小問的解答。
            \end{solution}

            \item 第二個小問。

            \begin{solution}
                第二個小問的解答。
            \end{solution}

        \end{enumerate}

        \item 第三個問題。

        \begin{solution}
            第三個問題的解答。
        \end{solution}

    \end{enumerate}
    在末尾使用 \cverb|\noqed|(或 \cverb|\noQED|)可以取消用於標記當前問題結束的方框。
    \noQED
\end{problem}


\bigskip\textcolor{gray!55}{(如果你希望手動編號一個習題…)}

\ManualNumbering{exercise}{A.1.1}
\begin{exercise}[一個手動編號的習題]
    使用 \cverb|\ManualNumbering| 以手動對某個習題進行編號。 這個編號只會影響下一個被指定的環境。
\end{exercise}

\begin{exercise}
    之後編號便會恢復正常。
\end{exercise}


\bigskip\textcolor{gray!55}{(如果你有暫時解決不出來的問題…)}

\DNF<一些描述>


\end{document}
