\documentclass[11pt,
  logo = {example-image},
  % logo height = 1cm,
  title in boldface,
  % title in sffamily,
  theorem in new line,
  % twoside,
]{homework}

\UseLanguage{Chinese}


\title{学科名称,第一周}
\author{作者}
% \date{\today,地点}
% \date{\today[only-year-month],地点}
\date{\TheDate{2023-12-25},地点}


\begin{document}


\bigskip\textcolor{gray!55}{(如果你打算直接写出解答…)}

\begin{problem}
    这里是解答/证明。
\end{problem}


\bigskip\textcolor{gray!55}{(如果你打算先陈述问题后再写出解答…)}

\begin{problem}[问题简介]
    你可以把问题陈述在这里…
\end{problem}

\begin{solution}
    …然后把解答写在这里…
\end{solution}

\bigskip\textcolor{gray!55}{(如果比起\textquote{解},你更希望写\textquote{证明}…)}

\begin{solution}[证明]
    …或者写一个这样的证明…
    \begin{lemma}[你可以在这里写一些注释]\label{lem}
        一些辅助结果。
    \end{lemma}
    \begin{proof}
        \Cref{lem}的证明,其中用到下面的公式:
        \[
            \infty = \infty + 1
            \makebox[0pt][l]{\,。} % 句尾的句号
            \qedhere               % 用来把 QED 放在正确的位置
        \]
    \end{proof}
    …剩余的步骤……
\end{solution}

\bigskip\textcolor{gray!55}{(写\textquote{答}也是可以的…)}

\begin{answer}
    \verb|answer| 环境的用法和 \verb|solution| 是完全相同的。
\end{answer}


\bigskip\textcolor{gray!55}{(如果你更喜欢传统的证明样式…)}

\begin{proof}
    \verb|proof| 环境依然可用。
\end{proof}


\DNF<一些描述>


\end{document}
