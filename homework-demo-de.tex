\documentclass[11pt,
  logo = {example-image},
  % logo height = 1cm,  % logo width = 2cm,
  title in boldface,
  % title in sffamily,
  theorem in new line,
  % twoside,
]{homework}

%% For highlighting the code in this document
\usepackage{newverbs}
\newverbcommand{\cverb}{\color{red!50!orange}}{}

%% Below is the main document


\UseLanguage{German}


\title{Das Thema, Woche 1}
\author{Vorname NAME}
% \date{\today, der Standort}
% \date{\today[only-year-month], der Standort}
\date{\TheDate{2023-12-25}, der Standort}


\begin{document}


\bigskip\textcolor{gray!55}{(Wenn Sie die Antwort direkt schreiben möchten ...)}

\begin{problem}
    Hier liegt die Lösung / der Beweis.
\end{problem}


\bigskip\textcolor{gray!55}{(Wenn Sie das Problem angeben und dann Ihre Antwort schreiben möchten ...)}

\begin{problem}[Kurze Beschreibung]
    Sie können das Problem auch hier angeben ...
\end{problem}

\begin{solution}
    ... und schreibe hier die Lösung ...
\end{solution}

\bigskip\textcolor{gray!55}{(Wenn Sie \textquote{Beweis} anstelle von \textquote{Lösung} bevorzugen ...)}

\begin{solution}[Beweis]
    ... oder ein Beweis wie dieser ...
    \begin{lemma}[Sie können hier eine Beschreibung schreiben]\label{lem}
        Einige zusätzlich Ergebnisse.
    \end{lemma}
    \begin{proof}
        Der Beweis \cref[gen.]{lem}, wobei wir die folgende Formel verwenden (beachten Sie die Verwendung von \cverb|\qedhere|): % oder der Beweis \cref[von]{lem}, oder der Beweis \cref[von, dat.]{lem}
        \[
            \infty = \infty + 1.
            \qedhere % Um das Q.E.D.-Symbol an der richtigen Stelle zu stellen.
        \]
    \end{proof}
    \begin{fact}[Für dieses Ergebnis ist kein Beweis erforderlich]
        \proofless
        Verwenden Sie \cverb|\proofless| um den hohlen Kasten, der das Ende einer Theorem-artigen Umgebung markiert, in einen festen Kasten umwandeln.
    \end{fact}
    ... und die übrigen Schritte ...
\end{solution}


\bigskip\textcolor{gray!55}{(Wenn Sie möchten, können Sie auch \texttt{answer} anstelle von \texttt{solution} schreiben ...)}

\begin{answer}
    Die Verwendung der \verb|answer|-Umgebung ist genau gleich wie \verb|solution|.
\end{answer}


\bigskip\textcolor{gray!55}{(Wenn Sie den klassischen Proof-Stil bevorzugen ...)}

\begin{proof}
    Die übliche \verb|proof|-Umgebung funktioniert ebenfalls.
\end{proof}


\bigskip\textcolor{gray!55}{(Wenn Sie jede Unterfrage eines Problems einzeln beantworten möchten ...)}

\begin{problem}[Ein Problem mit vielen Unterfragen]
    \begin{enumerate}[itemsep=.5\baselineskip]
        \item Die erste Frage.

        \begin{solution}
            Die Lösung der ersten Frage.
        \end{solution}

        \item Die zweite Frage.

        \begin{enumerate}[itemsep=.3\baselineskip]
            \item Die erste Unterfrage.

            \begin{solution}
                Die Lösung der ersten Unterfrage.
            \end{solution}

            \item Die zweite Unterfrage.

            \begin{solution}
                Die Lösung der zweiten Unterfrage.
            \end{solution}

        \end{enumerate}

        \item Die dritte Frage.

        \begin{solution}
            Die Lösung der dritten Frage.
        \end{solution}

    \end{enumerate}
    Verwenden Sie \cverb|\noqed| (oder \cverb|\noQED|) am Ende, um das Q.E.D.-Symbol zu unterdrücken, das das Ende des aktuellen Problems markiert.
    \noQED
\end{problem}


\bigskip\textcolor{gray!55}{(Wenn es eine Frage gibt, für die Sie im Moment keine Lösung finden können ...)}

\DNF<etwas Beschreibung>


\end{document}
