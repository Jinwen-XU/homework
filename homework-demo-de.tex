\documentclass[11pt,
  logo = {example-image},
  % logo height = 1cm,
  title in boldface,
  % title in sffamily,
  theorem in new line,
  % twoside,
]{homework}

\UseLanguage{German}


\title{Das Thema, Woche 1}
\author{Vorname NAME}
% \date{\today, der Standort}
% \date{\today[only-year-month], der Standort}
\date{\TheDate{2023-12-25}, der Standort}


\begin{document}


\bigskip\textcolor{gray!55}{(Wenn Sie die Antwort direkt schreiben möchten ...)}

\begin{problem}
    Hier liegt die Lösung / der Beweis.
\end{problem}


\bigskip\textcolor{gray!55}{(Wenn Sie das Problem angeben und dann Ihre Antwort schreiben möchten ...)}

\begin{problem}[Kurze Beschreibung]
    Sie können das Problem auch hier angeben ...
\end{problem}

\begin{solution}
    ... und schreibe hier die Lösung ...
\end{solution}

\bigskip\textcolor{gray!55}{(Wenn Sie \textquote{Beweis} anstelle von \textquote{Lösung} bevorzugen ...)}

\begin{solution}[Beweis]
    ... oder ein Beweis wie dieser ...
    \begin{lemma}[Sie können hier eine Beschreibung schreiben]\label{lem}
        Einige zusätzlich Ergebnisse.
    \end{lemma}
    \begin{proof}
        Der Beweis \cref[gen.]{lem}, wobei wir die folgende Formel verwenden: % oder der Beweis \cref[von]{lem}, oder der Beweis \cref[von, dat.]{lem}
        \[
            \infty = \infty + 1.
            \qedhere % Um das QED-Symbol an der richtigen Stelle zu stellen
        \]
    \end{proof}
    ... und die übrigen Schritte ...
\end{solution}


\bigskip\textcolor{gray!55}{(Wenn Sie möchten, können Sie auch \texttt{answer} anstelle von \texttt{solution} schreiben ...)}

\begin{answer}
    Die Verwendung der \verb|answer|-Umgebung ist genau gleich wie \verb|solution|.
\end{answer}


\bigskip\textcolor{gray!55}{(Wenn Sie den klassischen Proof-Stil bevorzugen ...)}

\begin{proof}
    Die übliche \verb|proof|-Umgebung funktioniert ebenfalls.
\end{proof}


\DNF<etwas Beschreibung>


\end{document}
