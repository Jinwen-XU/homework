\documentclass[11pt,
  logo = {example-image},
  % logo height = 1cm,
  title in boldface,
  % title in sffamily,
  theorem in new line,
]{homework}


\title{The Subject, Week 1}
\author{Author NAME}
% \date{\today[only-year-month], Location}
\date{\TheDate{2023-12-25}, Location}


\begin{document}


\bigskip\textcolor{gray!55}{(If you wish to write the answer directly...)}

\begin{problem}
    Here lies the solution / proof.
\end{problem}


\bigskip\textcolor{gray!55}{(If you wish to state the problem and then write your answer...)}

\begin{problem}[Brief description]
    You may also state the problem here...
\end{problem}

\begin{solution}
    ... and write the solution here...
\end{solution}

\bigskip\textcolor{gray!55}{(If you prefer \textquote{Proof} instead of \textquote{Solution}...)}

\begin{solution}[Proof]
    ... or a proof like this one...
    \begin{lemma}[You may write some description here]\label{lem}
        Some auxiliary result.
    \end{lemma}
    \begin{proof}
        The proof of \cref{lem}, where we use the following formula:
        \[
            \infty = \infty + 1.
            \qedhere % To place the QED symbol in the right place
        \]
    \end{proof}
    ... and the rest steps...
\end{solution}


\bigskip\textcolor{gray!55}{(You may also write \texttt{answer} instead of \texttt{solution} if you wish...)}

\begin{answer}
    The usage of the \verb|answer| environment is exactly the same as \verb|solution|.
\end{answer}


\bigskip\textcolor{gray!55}{(If you prefer the classical proof style...)}

\begin{proof}
    The usual \verb|proof| environment also works.
\end{proof}


\DNF<some description>


\end{document}
