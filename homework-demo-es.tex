\documentclass[11pt,
  logo = {example-image},
  % logo height = 1cm,
  title in boldface,
  % title in sffamily,
  theorem in new line,
  % twoside,
]{homework}

\UseLanguage{Spanish}


\title{El tema, semana 1}
\author{Nombre APELLIDO}
% \date{\today, Ubicación}
% \date{\today[only-year-month], Ubicación}
\date{\TheDate{2023-12-25}, Ubicación}


\begin{document}


\bigskip\textcolor{gray!55}{(Si desea escribir la respuesta directamente...)}

\begin{problem}
    Aquí está la solución / la prueba.
\end{problem}


\bigskip\textcolor{gray!55}{(Si desea plantear el problema y luego escribir su respuesta...)}

\begin{problem}[Breve descripción]
    También puedes plantear el problema aquí...
\end{problem}

\begin{solution}
    ... y escribir la solución aquí...
\end{solution}

\bigskip\textcolor{gray!55}{(Si prefiere \textquote{Prueba} en lugar de \textquote{Solución}...)}

\begin{solution}[Prueba]
    ... o una prueba como ésta...
    \begin{lemma}[Puedes escribir alguna descripción aquí]\label{lem}
        Algún resultado auxiliar.
    \end{lemma}
    \begin{proof}
        La prueba \cref[de]{lem}, donde usamos la siguiente fórmula:
        \[
            \infty = \infty + 1.
            \qedhere % Para colocar el símbolo QED en el lugar correcto.
        \]
    \end{proof}
    ... y el resto pasos...
\end{solution}


\bigskip\textcolor{gray!55}{(También puedes escribir \texttt{answer} en lugar de \texttt{solution} si lo deseas...)}

\begin{answer}
    El uso del entorno \verb|answer| es exactamente lo mismo que \verb|solution|.
\end{answer}


\bigskip\textcolor{gray!55}{(Si prefieres el estilo clásico de prueba...)}

\begin{proof}
    El entorno habitual de \verb|proof| también funciona.
\end{proof}


\DNF<alguna descripción>


\end{document}
