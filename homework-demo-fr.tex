\documentclass[11pt,
  logo = {example-image},
  % logo height = 1cm,
  title in boldface,
  % title in sffamily,
  theorem in new line,
]{homework}

\UseLanguage{French}


\title{Le sujet, semaine 1}
\author{Prénom NOM}
% \date{\today[only-year-month], Location}
\date{\TheDate{2023-12-25}, Location}


\begin{document}


\bigskip\textcolor{gray!55}{(Si vous souhaitez écrire la réponse directement...)}

\begin{problem}
    Voici la solution / la preuve.
\end{problem}


\bigskip\textcolor{gray!55}{(Si vous souhaitez énoncer le problème puis écrire votre réponse...)}

\begin{problem}[Description brève]
    Vous pouvez également énoncer le problème ici...
\end{problem}

\begin{solution}
    ... puis écrire la solution ici...
\end{solution}

\bigskip\textcolor{gray!55}{(Si vous préférez \textquote{Preuve} au lieu de \textquote{Solution}...)}

\begin{solution}[Preuve]
    ... ou une preuve comme ça...
    \begin{lemma}[Vous pouvez écrire ici quelques descriptions]\label{lem}
        Un résultat auxiliaire.
    \end{lemma}
    \begin{proof}
        La preuve \cref[de]{lem}, où l'on utilise la formule suivante :
        \[
            \infty = \infty + 1.
            \qedhere % pour placer le symbole QED au bon endroit
        \]
    \end{proof}
    ... et les étapes restantes...
\end{solution}


\bigskip\textcolor{gray!55}{(Vous pouvez également écrire \texttt{answer} au lieu de \texttt{solution} si vous le souhaitez...)}

\begin{answer}
    L'utilisation de l'environnement \verb|answer| est exactement le même que celui de \verb|solution|.
\end{answer}


\bigskip\textcolor{gray!55}{(Si vous préférez le style classique...)}

\begin{proof}
    L'environnement habituel \verb|proof| fonctionne également.
\end{proof}


\DNF<la description>


\end{document}
