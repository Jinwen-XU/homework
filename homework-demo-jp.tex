\documentclass[11pt,
  logo = {example-image},
  % logo height = 1cm,
  title in boldface,
  % title in sffamily,
  theorem in new line,
  % twoside,
]{homework}

\UseLanguage{Japanese}


\title{科目名、第 1 週}
\author{著者名}
% \date{\today、所在地}
% \date{\today[only-year-month]、所在地}
\date{\TheDate{2023-12-25}、所在地}


\begin{document}


\bigskip\textcolor{gray!55}{(ソリューションを直接書くつもりなら…)}

\begin{problem}
    これが答え/証明です。
\end{problem}


\bigskip\textcolor{gray!55}{(もし、問題を述べてからソリューションを書きたいのであれば...)}

\begin{problem}[問題紹介]
    ここに問題を述べて…
\end{problem}

\begin{solution}
    …ここにソリューションを書くことができます...
\end{solution}

\bigskip\textcolor{gray!55}{(もし\textquote{解答}よりも\textquote{証明}を書きたいのなら...)}

\begin{solution}[証明]
    …または次のような証明を書きます...
    \begin{lemma}[ここにコメントを書くことができます]\label{lem}
        いくつかの補助的な結果。
    \end{lemma}
    \begin{proof}
        これが\cref{lem}の証明です。次の式が使用されます:
        \[
            \infty = \infty + 1
            \makebox[0pt][l]{\,。} % 文末のピリオド
            \qedhere               % QED シンボルを正しい位置に配置するために使用されます
        \]
    \end{proof}
    …そして残りのステップ…
\end{solution}

\bigskip\textcolor{gray!55}{(\texttt{answer} 環境を使用することもできます…)}

\begin{answer}
    \verb|answer| 環境は \verb|solution| 環境とまったく同じように使用されます。
\end{answer}


\bigskip\textcolor{gray!55}{(伝統的な証明スタイルがお好みなら…)}

\begin{proof}
    \verb|proof| 環境はまだ利用可能です。
\end{proof}


\DNF<いくつかの説明>


\end{document}
