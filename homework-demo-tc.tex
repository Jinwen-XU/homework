\documentclass[11pt,
  logo = {example-image},
  % logo height = 1cm,
  title in boldface,
  % title in sffamily,
  theorem in new line,
  % twoside,
]{homework}

\UseLanguage{Traditional Chinese}


\title{學科名稱,第一周}
\author{作者}
% \date{\today,地點}
% \date{\today[only-year-month],地點}
\date{\TheDate{2023-12-25},地點}


\begin{document}


\bigskip\textcolor{gray!55}{(如果你打算直接寫出解答…)}

\begin{problem}
    這裡是解答/證明。
\end{problem}


\bigskip\textcolor{gray!55}{(如果你打算先陳述問題後再寫出解答…)}

\begin{problem}[問題簡介]
    你可以把問題陳述在這裡…
\end{problem}

\begin{solution}
    …然後把解答寫在這裡…
\end{solution}

\bigskip\textcolor{gray!55}{(如果比起\textquote{解},你更希望寫\textquote{證明}…)}

\begin{solution}[證明]
    …或者寫一個這樣的證明…
    \begin{lemma}[你可以在這裡寫一些註釋]\label{lem}
        一些輔助結果。
    \end{lemma}
    \begin{proof}
        \Cref{lem}的證明,其中用到下面的公式:
        \[
            \infty = \infty + 1
            \makebox[0pt][l]{\,。} % 句尾的句號
            \qedhere               % 用來把 QED 放在正確的位置
        \]
    \end{proof}
    …剩餘的步驟…
\end{solution}

\bigskip\textcolor{gray!55}{(寫\textquote{答}也是可以的…)}

\begin{answer}
    \verb|answer| 環境的用法和 \verb|solution| 是完全相同的。
\end{answer}


\bigskip\textcolor{gray!55}{(如果你更喜歡傳統的證明樣式…)}

\begin{proof}
    \verb|proof| 環境仍可使用。
\end{proof}


\DNF<一些描述>


\end{document}
